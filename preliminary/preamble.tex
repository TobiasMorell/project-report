\documentclass[12pt,a4paper,openright]{report}
%%%%%%%%%%%%%%%%%%%%%%%%%%%%%%%%%%%%%%%%%%%%%%%%
% Language, Encoding and Fonts
% http://en.wikibooks.org/wiki/LaTeX/Internationalization
%%%%%%%%%%%%%%%%%%%%%%%%%%%%%%%%%%%%%%%%%%%%%%%%
% Select encoding of your inputs. Depends on
% your operating system and its default input
% encoding. Typically, you should use
%   Linux  : utf8 (most modern Linux distributions)
%            latin1
%   Windows: ansinew
%            latin1 (works in most cases)
%   Mac    : applemac
% Notice that you can manually change the input
% encoding of your files by selecting "save as"
% an select the desired input encoding.
\usepackage{scrextend}
\usepackage[utf8]{inputenc}
%\usepackage{float}
\usepackage{floatrow}
\usepackage{booktabs}
\usepackage{multicol}
\usepackage{siunitx}
\usepackage{csvsimple}
\usepackage[table,xcdraw]{xcolor}
% Make latex understand and use the typographic
% rules of the language used in the document.
\usepackage{array}
\usepackage[english]{babel}
% Use the vector font Latin Modern which is going
% to be the default font in latex in the future.
\usepackage{lmodern}
% Choose the font encoding
\usepackage[T1]{fontenc}
\usepackage{lastpage} % Tilføjet af Jesper, får sidetal til at virke.
%%%%%%%%%%%%%%%%%%%%%%%%%%%%%%%%%%%%%%%%%%%%%%%%
% Graphics and Tables
% http://en.wikibooks.org/wiki/LaTeX/Importing_Graphics
% http://en.wikibooks.org/wiki/LaTeX/Tables
% http://en.wikibooks.org/wiki/LaTeX/Colors
%%%%%%%%%%%%%%%%%%%%%%%%%%%%%%%%%%%%%%%%%%%%%%%%
% load a colour package
\usepackage[table]{xcolor}

\definecolor{aaublue}{RGB}{33,26,82}% dark blue
\definecolor{editgreen}{HTML}{008000}% dark green
% The standard graphics inclusion package
\usepackage{graphicx}
% Set up how figure and table captions are displayed
\usepackage[within=none]{caption}
\captionsetup{%
  font=footnotesize,% set font size to footnotesize
  labelfont=bf % bold label (e.g., Figure 3.2) font
}
% Make the standard latex tables look so much better
\usepackage{array,booktabs}
% Enable the use of frames around, e.g., theorems
% The framed package is used in the example environment
\usepackage{framed}

\usepackage{tikz}
\usetikzlibrary{arrows,arrows.meta, fit, backgrounds, positioning, automata, shapes}
\usepackage{pgfplots}
\pgfplotsset{compat=1.15}
\usepackage{enumitem}
%%%%%%%%%%%%%%%%%%%%%%%%%%%%%%%%%%%%%%%%%%%%%%%%
% Mathematics
% http://en.wikibooks.org/wiki/LaTeX/Mathematics
%%%%%%%%%%%%%%%%%%%%%%%%%%%%%%%%%%%%%%%%%%%%%%%%
% Defines new environments such as equation,
% align and split
\usepackage{amsmath}
% Adds new math symbols
\usepackage{amssymb}
% Use theorems in your document
% The ntheorem package is also used for the example environment
% When using thmmarks, amsmath must be an option as well. Otherwise \eqref doesn't work anymore.
\usepackage[framed,amsmath,thmmarks]{ntheorem}

%%%%%%%%%%%%%%%%%%%%%%%%%%%%%%%%%%%%%%%%%%%%%%%%
% Page Layout
% http://en.wikibooks.org/wiki/LaTeX/Page_Layout
%%%%%%%%%%%%%%%%%%%%%%%%%%%%%%%%%%%%%%%%%%%%%%%%
% Change margins, papersize, etc of the document
\usepackage[top=1in, bottom=1.5in]{geometry}%, left=1in, right=1in]{geometry}

\setlength\parindent{0pt}   % Remove paragraph indentation
\setlength\parskip{1em}     % Sets paragraph spacing
%\usepackage[
%  inner=28mm,% left margin on an odd page
%  outer=41mm,% right margin on an odd page
%  ]{geometry}
% Modify how \chapter, \section, etc. look
% The titlesec package is very configureable
\usepackage{titlesec}
\titleformat*{\section}{\normalfont\Large\bfseries\color{aaublue}} % Do we REALLY want blue titles?
\titleformat*{\subsection}{\normalfont\large\bfseries\color{aaublue}}
\titleformat*{\subsubsection}{\normalfont\normalsize\bfseries\color{aaublue}}
%\titleformat*{\paragraph}{\normalfont\normalsize\bfseries\color{aaublue}}
%\titleformat*{\subparagraph}{\normalfont\normalsize\bfseries\color{aaublue}}

% Clear empty pages between chapters
\let\origdoublepage\cleardoublepage
\newcommand{\clearemptydoublepage}{%
  \clearpage
  {\pagestyle{empty}\origdoublepage}%
}
\let\cleardoublepage\clearemptydoublepage

% Change the headers and footers
\usepackage{fancyhdr}
\pagestyle{fancy}
\fancyhf{} %delete everything
\renewcommand{\headrulewidth}{0pt} %remove the horizontal line in the header
%\fancyhead[R]{\color{aaublue}\small\nouppercase\leftmark} %even page - chapter title
%\fancyhead[LO]{\color{aaublue}\small\nouppercase\rightmark} %uneven page - section title
\fancyfoot[C]{\thepage} %page number on all pages
\setlength{\headheight}{14pt}
% Do not stretch the content of a page. Instead,
% insert white space at the bottom of the page
\raggedbottom
% Enable arithmetics with length. Useful when
% typesetting the layout.
\usepackage{calc}

\usepackage[
%  disable, %turn off todonotes
  colorinlistoftodos, %enable a coloured square in the list of todos
  textwidth=1.1\marginparwidth, %set the width of the todonotes
  textsize=scriptsize, %size of the text in the todonotes
  ]{todonotes}

\usepackage[T1]{fontenc}
\usepackage{inconsolata}

\usepackage{color}


\usepackage{listings}
%%%%%%%%%%%% COLOURS
\definecolor{bluekeywords}{rgb}{0.13,0.13,1}
\definecolor{greencomments}{rgb}{0,0.5,0}
\definecolor{redstrings}{rgb}{0.9,0,0}
\definecolor{greyannotations}{rgb}{0.46,0.45,0.48}
\definecolor{mGreen}{rgb}{0,0.6,0}
\definecolor{mGray}{rgb}{0.5,0.5,0.5}
\definecolor{mPurple}{rgb}{0.58,0,0.82}

%%% This should not set basicstyle
\lstset{
    basicstyle=\ttfamily\scriptsize, 
    xleftmargin = -15pt,
    numbers = left,
    numbersep = -35pt,
    frame=single,
    framexleftmargin=15pt,
    captionpos=b
}


%%%%%%%%%%%% STYLES
\lstdefinestyle{java-highlight}{
    basicstyle=\ttfamily\scriptsize, 
    showspaces=false,
    showtabs=false,
    breaklines=true,
    showstringspaces=false,
    breakatwhitespace=true,
    keywordstyle=\color{bluekeywords},
    commentstyle=\color{greencomments},
    stringstyle=\color{redstrings},
}

\lstdefinestyle{haskell-highlight}{
  xleftmargin=2pt,
  stepnumber=1,
  numbers=left,
  numbersep=5pt,
  numberstyle=\ttfamily\tiny\color[gray]{0.3},
  belowcaptionskip=\bigskipamount,
  captionpos=b,
  escapeinside={*'}{'*},
  language=Haskell,
  tabsize=2,
  emphstyle={\bf},
  commentstyle=\it,
  stringstyle=\mdseries\rmfamily,
  showspaces=false,
  keywordstyle=\bfseries\rmfamily,
  columns=flexible,
  basicstyle=\small\sffamily,
  showstringspaces=false,
  morecomment=[l]\%,
  }

% default style
\lstdefinelanguage{csharp}{
    basicstyle=\ttfamily\scriptsize, 
    showspaces=false,
    showtabs=false,
    breaklines=true,
    showstringspaces=false,
    breakatwhitespace=true,
    escapeinside={(*@}{@*)},
    commentstyle=\color{greencomments},
    morekeywords={partial, var, value, get, set, static, async, await, foreach, in, ref, new, string},
    keywordstyle=\color{bluekeywords},
    stringstyle=\color{redstrings}
}

\lstdefinelanguage{fsharp}{
    basicstyle=\ttfamily\scriptsize,
    morekeywords={let, new, match, with, rec, open, module, namespace, type, of, member, and, for, while, true, false, in, do, begin, end, fun, function, return, yield, try, mutable, if, then, else, cloud, async, static, use, abstract, interface, inherit, finally },
    otherkeywords={ let!, return!, do!, yield!, use!, var, from, select, where, order, by },
    keywordstyle=\color{bluekeywords},
    sensitive=true,
	breaklines=true,
    xleftmargin=-1cm,
    xrightmargin=-1cm,
    aboveskip=\bigskipamount,
	tabsize=4,
    morecomment=[l][\color{greencomments}]{///},
    morecomment=[l][\color{greencomments}]{//},
    morecomment=[s][\color{greencomments}]{{(*}{*)}},
    morestring=[b]",
    showstringspaces=false,
    literate={`}{\`}1,
    stringstyle=\color{redstrings},
}
\lstdefinelanguage{PEG}{
    basicstyle=\ttfamily\scriptsize,
    keywords={=, /, \{, \}},
    keywordstyle=\bfseries,
    identifierstyle=\color{black},
    sensitive=false,
    comment=[l]{//},
    morecomment=[s]{/*}{*/},
    commentstyle=\color{purple}\ttfamily,
    stringstyle=\color{red}\ttfamily,
    morestring=[b]',
    morestring=[b]",
    tabsize=4,
    breaklines=true,
}
\lstdefinelanguage{JavaScript}{
    basicstyle=\ttfamily\scriptsize,
    keywords={typeof, new, true, false, catch, function, return, null, catch, switch, var, const, if, in, while, do, else, case, break},
    keywordstyle=\color{blue}\bfseries,
    ndkeywords={class, export, boolean, throw, implements, import, this},
    ndkeywordstyle=\color{darkgray}\bfseries,
    identifierstyle=\color{black},
    sensitive=false,
    comment=[l]{//},
    morecomment=[s]{/*}{*/},
    commentstyle=\color{purple}\ttfamily,
    stringstyle=\color{red}\ttfamily,
    morestring=[b]',
    morestring=[b]"
}
\lstdefinelanguage{HTML5}{
    language=html,
    sensitive=true,	
    alsoletter={<>=-},	
    morecomment=[s]{<!-}{-->},
    tag=[s],
    otherkeywords={
    % General
    >,
    % Standard tags
    <!DOCTYPE,
    </html, <html, <head, <title, </title, <style, </style, <link, </head, <meta, />,
    % body
    </body, <body,
    % Divs
    </div, <div, </div>, 
    % Paragraphs
    </p, <p, </p>,
    % scripts
    </script, <script,
    % More tags...
    <canvas, /canvas>, <svg, <rect, <animateTransform, </rect>, </svg>, <video, <source, <iframe, </iframe>, </video>, <image, </image>, <header, </header, <article, </article
    },
    ndkeywords={
    % General
    =,
    % HTML attributes
    charset=, src=, id=, width=, height=, style=, type=, rel=, href=,
    % SVG attributes
    fill=, attributeName=, begin=, dur=, from=, to=, poster=, controls=, x=, y=, repeatCount=, xlink:href=,
    % properties
    margin:, padding:, background-image:, border:, top:, left:, position:, width:, height:, margin-top:, margin-bottom:, font-size:, line-height:,
    % CSS3 properties
    transform:, -moz-transform:, -webkit-transform:,
    animation:, -webkit-animation:,
    transition:,  transition-duration:, transition-property:, transition-timing-function:,
}
}

\lstdefinelanguage{Haskellana}{%
    language     = Haskell,
    morekeywords = {fix},
}

\lstdefinelanguage{clojure}%
{morekeywords={*,*1,*2,*3,*agent*,*allow-unresolved-vars*,*assert*,*clojure-version*,*command-line-args*,%
*compile-files*,*compile-path*,*e,*err*,*file*,*flush-on-newline*,*in*,*macro-meta*,%
*math-context*,*ns*,*out*,*print-dup*,*print-length*,*print-level*,*print-meta*,*print-readably*,%
*read-eval*,*source-path*,*use-context-classloader*,*warn-on-reflection*,+,-,->,->>,..,/,:else,%
<,<=,=,==,>,>=,@,accessor,aclone,add-classpath,add-watch,agent,agent-errors,aget,alength,alias,%
all-ns,alter,alter-meta!,alter-var-root,amap,ancestors,and,apply,areduce,array-map,aset,%
aset-boolean,aset-byte,aset-char,aset-double,aset-float,aset-int,aset-long,aset-short,assert,%
assoc,assoc!,assoc-in,associative?,atom,await,await-for,await1,bases,bean,bigdec,bigint,binding,%
bit-and,bit-and-not,bit-clear,bit-flip,bit-not,bit-or,bit-set,bit-shift-left,bit-shift-right,%
bit-test,bit-xor,boolean,boolean-array,booleans,bound-fn,bound-fn*,butlast,byte,byte-array,%
bytes,cast,char,char-array,char-escape-string,char-name-string,char?,chars,chunk,chunk-append,%
chunk-buffer,chunk-cons,chunk-first,chunk-next,chunk-rest,chunked-seq?,class,class?,%
clear-agent-errors,clojure-version,coll?,comment,commute,comp,comparator,compare,compare-and-set!,%
compile,complement,concat,cond,condp,conj,conj!,cons,constantly,construct-proxy,contains?,count,%
counted?,create-ns,create-struct,cycle,dec,decimal?,declare,def,definline,defmacro,defmethod,%
defmulti,defn,defn-,defonce,defprotocol,defstruct,deftype,delay,delay?,deliver,deref,derive,%
descendants,destructure,disj,disj!,dissoc,dissoc!,distinct,distinct?,do,do-template,doall,doc,%
dorun,doseq,dosync,dotimes,doto,double,double-array,doubles,drop,drop-last,drop-while,empty,empty?,%
ensure,enumeration-seq,eval,even?,every?,false,false?,ffirst,file-seq,filter,finally,find,find-doc,%
find-ns,find-var,first,float,float-array,float?,floats,flush,fn,fn?,fnext,for,force,format,future,%
future-call,future-cancel,future-cancelled?,future-done?,future?,gen-class,gen-interface,gensym,%
get,get-in,get-method,get-proxy-class,get-thread-bindings,get-validator,hash,hash-map,hash-set,%
identical?,identity,if,if-let,if-not,ifn?,import,in-ns,inc,init-proxy,instance?,int,int-array,%
integer?,interleave,intern,interpose,into,into-array,ints,io!,isa?,iterate,iterator-seq,juxt,%
key,keys,keyword,keyword?,last,lazy-cat,lazy-seq,let,letfn,line-seq,list,list*,list?,load,load-file,%
load-reader,load-string,loaded-libs,locking,long,long-array,longs,loop,macroexpand,macroexpand-1,%
make-array,make-hierarchy,map,map?,mapcat,max,max-key,memfn,memoize,merge,merge-with,meta,%
method-sig,methods,min,min-key,mod,monitor-enter,monitor-exit,name,namespace,neg?,new,newline,%
next,nfirst,nil,nil?,nnext,not,not-any?,not-empty,not-every?,not=,ns,ns-aliases,ns-imports,%
ns-interns,ns-map,ns-name,ns-publics,ns-refers,ns-resolve,ns-unalias,ns-unmap,nth,nthnext,num,%
number?,odd?,or,parents,partial,partition,pcalls,peek,persistent!,pmap,pop,pop!,pop-thread-bindings,%
pos?,pr,pr-str,prefer-method,prefers,primitives-classnames,print,print-ctor,print-doc,print-dup,%
print-method,print-namespace-doc,print-simple,print-special-doc,print-str,printf,println,println-str,%
prn,prn-str,promise,proxy,proxy-call-with-super,proxy-mappings,proxy-name,proxy-super,%
push-thread-bindings,pvalues,quot,rand,rand-int,range,ratio?,rational?,rationalize,re-find,%
re-groups,re-matcher,re-matches,re-pattern,re-seq,read,read-line,read-string,recur,reduce,ref,%
ref-history-count,ref-max-history,ref-min-history,ref-set,refer,refer-clojure,reify,%
release-pending-sends,rem,remove,remove-method,remove-ns,remove-watch,repeat,repeatedly,%
replace,replicate,require,reset!,reset-meta!,resolve,rest,resultset-seq,reverse,reversible?,%
rseq,rsubseq,second,select-keys,send,send-off,seq,seq?,seque,sequence,sequential?,set,set!,%
set-validator!,set?,short,short-array,shorts,shutdown-agents,slurp,some,sort,sort-by,sorted-map,%
sorted-map-by,sorted-set,sorted-set-by,sorted?,special-form-anchor,special-symbol?,split-at,%
split-with,str,stream?,string?,struct,struct-map,subs,subseq,subvec,supers,swap!,symbol,symbol?,%
sync,syntax-symbol-anchor,take,take-last,take-nth,take-while,test,the-ns,throw,time,to-array,%
to-array-2d,trampoline,transient,tree-seq,true,true?,try,type,unchecked-add,unchecked-dec,%
unchecked-divide,unchecked-inc,unchecked-multiply,unchecked-negate,unchecked-remainder,%
unchecked-subtract,underive,unquote,unquote-splicing,update-in,update-proxy,use,val,vals,%
var,var-get,var-set,var?,vary-meta,vec,vector,vector?,when,when-first,when-let,when-not,%
while,with-bindings,with-bindings*,with-in-str,with-loading-context,with-local-vars,%
with-meta,with-open,with-out-str,with-precision,xml-seq,zero?,zipmap
},%
   sensitive,% ???
   alsodigit=-,%
   morecomment=[l];,%
   morestring=[b]"%
  }[keywords,comments,strings]%

\lstdefinestyle{C}{
    commentstyle=\color{mGreen},
    keywordstyle=\color{magenta},
    numberstyle=\tiny\color{mGray},
    stringstyle=\color{mPurple},
    basicstyle=\footnotesize,
    breakatwhitespace=false,         
    breaklines=true,                 
    captionpos=b,                    
    keepspaces=true,                 
    numbers=left,                    
    numbersep=5pt,                  
    showspaces=false,                
    showstringspaces=false,
    showtabs=false,                  
    tabsize=2,
    language=C
}




\usepackage{courier}
\definecolor{gray}{rgb}{0.4,0.4,0.4}
\definecolor{darkblue}{rgb}{0.0,0.0,0.6}
\definecolor{cyan}{rgb}{0.0,0.6,0.6}


%\setmonofont{Consolas} %to be used with XeLaTeX or LuaLaTeX
\definecolor{bluekeywords}{rgb}{0,0,1}
\definecolor{greencomments}{rgb}{0,0.5,0}
\definecolor{redstrings}{rgb}{0.64,0.08,0.08}
\definecolor{xmlcomments}{rgb}{0.5,0.5,0.5}
\definecolor{types}{rgb}{0.17,0.57,0.68}

\usepackage{hyperref}
\hypersetup{%
	plainpages=false,%
	pdfauthor={SW701E17 - Tobias Morell, Thomas, Daniel Bolhuis, Malte Rosenbjerg Andersen \& Thor},%
	pdftitle={P9},%
	pdfsubject={Project Report},%
	bookmarksnumbered=true,%
	colorlinks,%
	citecolor=aaublue,%
	filecolor=aaublue,%
	linkcolor=aaublue,% you should probably change this to black before printing
	urlcolor=aaublue,%
	pdfstartview=FitH%
}

% Quotations package
\usepackage[autostyle]{csquotes}
% PDF package
\usepackage[final]{pdfpages}
\usepackage{acronym}
\usepackage{longtable}
\usepackage{tasks}
\usepackage{subcaption}
\addtokomafont{labelinglabel}{\sffamily\bfseries}
\usepackage{dirtytalk}

\usepackage{silence}

\usepackage[backend=biber, style=ieee, urldate=long]{biblatex} % biblatex - bibliography tool %
\bibliography{preliminary/bib}
\renewcommand*{\lstlistlistingname}{List of Listings}

\setlength{\headheight}{15pt}

\usepackage{rotating}
\usepackage{tablefootnote}

\begin{filecontents*}{benchmark-tab.csv}
Test,Cry CSharp (Editor),Cry C++ (Editor),Cry C++ (Standalone),Dotnet CSharp (debug),Dotnet CSharp (release),Godot CSharp (editor),Godot CSharp (release),Mono CSharp (debug),Mono CSharp (release),Unity CSharp (editor),Unity CSharp (release),Unreal C++ (editor),VisualC++ (debug),VisualC++ (release),WSL C++ (debug),WSL C++ (release),Arcadia
Scale Vector 2D,86.95794582,1018.670502,922.221222,38.20768833,6.742976904,67.52986431,71.7416811,32.38273859,26.84256315,49.04926062,34.47378874,3.169403,1157.680054,78.630505,168.100224,19.376309,3766.614014
Scale Vector 3D,85.89270115,930.331345,1001.464462,41.86224937,4.096797407,66.98613167,70.87862968,64.0030241,67.67254829,44.48920488,29.00312424,3.55251,1156.498947,79.298434,179.511271,19.742976,3698.211182
Multiply Vector 2D,82.77453899,905.565948,909.602394,21.95443749,5.571398735,68.56144905,71.65192127,29.85697269,28.30824375,31.14423037,24.33060646,3.820123,2130.445862,153.303452,287.514935,34.545763,4073.972168
Multiply Vector 3D,100.4223919,1146.82373,1160.358429,38.36730957,4.132973552,70.12160301,73.60248566,54.80506182,55.87666988,34.00476694,24.2743969,3.976601,2173.554993,154.841766,306.929245,34.072058,4206.462402
Translate Vector 2D,53.29652309,886.934357,890.70755,20.37241459,19.32539701,49.68834877,53.95591497,23.01784754,23.83990049,62.32574463,40.73523045,4.451444,1100.215607,77.743173,143.48649,20.160066,4154.245117
Translate Vector 3D,52.96351433,1102.207336,1173.044052,20.64691782,20.22571921,56.91459656,61.22503757,30.64299583,30.61949492,59.28759098,35.66885948,3.77521,1111.334915,78.247614,150.885286,20.881981,4194.556152
Subtract Vector 2D,98.46003532,1014.142303,1024.944992,22.39426255,5.439406931,68.97805214,71.03662014,36.33780479,34.09849882,57.81182051,40.5850482,4.249314,2189.26178,153.196812,309.275723,37.236857,4074.209961
Subtract Vector 3D,115.3355789,1334.180832,1294.719315,21.67409539,5.565602779,70.25182724,70.81814289,35.90900421,34.0504241,53.56592178,35.08020878,4.497688,2250.258179,161.352253,350.33514,39.635711,4226.749023
Length Vector 2D,49.37628508,572.698288,570.471954,18.49542141,4.115792513,26.69751406,29.97713327,26.61603928,25.93693852,55.27316093,48.16655159,3.656694,1079.586258,78.533316,132.561636,20.237803,4026.231445
Length Vector 3D,46.39502048,696.754074,676.607056,19.7453177,3.928655982,34.05391693,37.27663755,30.76211452,31.54877186,58.95939589,50.28742075,4.021047,1096.700592,78.056669,144.001379,19.871365,4054.490234
Dot Product 2D,47.29133129,966.006279,900.705032,17.96553969,4.282022119,30.0716114,34.47232962,12.61104167,9.674120545,45.619452,34.94383097,3.415283,2270.132904,159.312868,401.981831,36.608317,3999.504883
Dot Product 3D,52.45328188,1186.862411,1191.672745,17.39601851,3.833503127,32.91000128,35.88950634,24.280864,24.38810706,49.09858942,35.97127199,3.228273,2388.15033,159.725313,443.025589,38.101602,4231.833008
Array Allocation,815155.2734,53163.71094,51060.35156,51987.93213,54322.20215,716776.7969,705091.4453,692183.8672,696468.6719,1214730.781,320137.8125,1483.035049,24914.66553,1676.538773,4176.936646,349.747238,3451.04541
Prime,3955.852661,62730.70801,64547.50488,2003.083038,1126.654892,4184.587708,4327.935181,2243.278961,1443.416672,6378.071899,4596.665649,4.87445,55140.60303,27586.69067,27591.0022,2.972949,9044046
Sestoft,43.13329458,136.674242,129.17757,19.79778409,9.166597128,43.07381868,44.32321548,35.69456339,37.00475931,41.59214258,36.2664628,11.101028,78.69453,12.865073,41.54355,9.355648,1014820.063
\end{filecontents*}


\begin{filecontents*}{benchmark-graph.csv}
Engine,Scale Vector 2D,Scale Vector 3D,Multiply Vector 2D,Multiply Vector 3D,Translate Vector 2D,Translate Vector 3D,Subtract Vector 2D,Subtract Vector 3D,Length Vector 2D,Length Vector 3D,Dot Product 2D,Dot Product 3D,Prime,Array Allocation,Sestoft
Cry CSharp (editor),86.95794582,85.89270115,82.77453899,100.4223919,53.29652309,52.96351433,98.46003532,115.3355789,49.37628508,46.39502048,47.29133129,52.45328188,815155.2734,3955.852661,43.13329458
Cry C++ (editor),1018.670502,930.331345,905.565948,1146.82373,886.934357,1102.207336,1014.142303,1334.180832,572.698288,696.754074,966.006279,1186.862411,53163.71094,62730.70801,136.674242
Cry C++ (standalone),922.221222,1001.464462,909.602394,1160.358429,890.70755,1173.044052,1024.944992,1294.719315,570.471954,676.607056,900.705032,1191.672745,51060.35156,64547.50488,129.17757
Dotnet CSharp (debug),38.20768833,41.86224937,21.95443749,38.36730957,20.37241459,20.64691782,22.39426255,21.67409539,18.49542141,19.7453177,17.96553969,17.39601851,51987.93213,2003.083038,19.79778409
Dotnet CSharp (release),6.742976904,4.096797407,5.571398735,4.132973552,19.32539701,20.22571921,5.439406931,5.565602779,4.115792513,3.928655982,4.282022119,3.833503127,54322.20215,1126.654892,9.166597128
Godot CSharp (editor),67.52986431,66.98613167,68.56144905,70.12160301,49.68834877,56.91459656,68.97805214,70.25182724,26.69751406,34.05391693,30.0716114,32.91000128,716776.7969,4184.587708,43.07381868
Godot CSharp (release),71.7416811,70.87862968,71.65192127,73.60248566,53.95591497,61.22503757,71.03662014,70.81814289,29.97713327,37.27663755,34.47232962,35.88950634,705091.4453,4327.935181,44.32321548
Mono CSharp (debug),32.38273859,64.0030241,29.85697269,54.80506182,23.01784754,30.64299583,36.33780479,35.90900421,26.61603928,30.76211452,12.61104167,24.280864,692183.8672,2243.278961,35.69456339
Mono CSharp (release),26.84256315,67.67254829,28.30824375,55.87666988,23.83990049,30.61949492,34.09849882,34.0504241,25.93693852,31.54877186,9.674120545,24.38810706,696468.6719,1443.416672,37.00475931
Unity CSharp (editor),49.04926062,44.48920488,31.14423037,34.00476694,62.32574463,59.28759098,57.81182051,53.56592178,55.27316093,58.95939589,45.619452,49.09858942,1214730.781,6378.071899,41.59214258
Unity CSharp (release),34.47378874,29.00312424,24.33060646,24.2743969,40.73523045,35.66885948,40.5850482,35.08020878,48.16655159,50.28742075,34.94383097,35.97127199,320137.8125,4596.665649,36.2664628
Unreal C++ (editor),3.169403,3.55251,3.820123,3.976601,4.451444,3.77521,4.249314,4.497688,3.656694,4.021047,3.415283,3.228273,1483.035049,4.87445,11.101028
Visual C++ (debug),1157.680054,1156.498947,2130.445862,2173.554993,1100.215607,1111.334915,2189.26178,2250.258179,1079.586258,1096.700592,2270.132904,2388.15033,24914.66553,55140.60303,78.69453
Visual C++ (release),78.630505,79.298434,153.303452,154.841766,77.743173,78.247614,153.196812,161.352253,78.533316,78.056669,159.312868,159.725313,1676.538773,27586.69067,12.865073
GCC C++ (debug),168.100224,179.511271,287.514935,306.929245,143.48649,150.885286,309.275723,350.33514,132.561636,144.001379,401.981831,443.025589,4176.936646,27591.0022,41.54355
GCC C++ (release),19.376309,19.742976,34.545763,34.072058,20.160066,20.881981,37.236857,39.635711,20.237803,19.871365,36.608317,38.101602,349.747238,2.972949,9.355648
Arcadia,3766.614014,3698.211182,4073.972168,4206.462402,4154.245117,4194.556152,4074.209961,4226.749023,4026.231445,4054.490234,3999.504883,4231.833008,3451.04541,9044046,1014820.063
\end{filecontents*}

\begin{filecontents*}{macro-benchmark-time.csv}
Iteration No.,Unity,Unreal
1,24575,49.9
2,1007.7,63.5
3,499.2,40.6
4,1300.7,25.2
5,172.1,54.6
6,206.2,49.9
7,711.9,152.5
8,935.9,31.2
9,128.7,28.9
10,177.8,65.8
11,146.5,23.3
12,189.4,40.1
13,1390.7,804.8
14,204.3,71.4
15,154.5,33.1
16,341.5,75.6
17,245.9,64.4
18,198.3,83.5
19,150.6,70.9
20,280.4,252.4
21,237.9,50.8
22,187.6,49
23,158.1,72.8
24,128.3,35.9
25,162.8,43.4
26,211.8,646.1
27,238.4,32.6
28,210.9,83.5
29,261.7,49.5
30,218.4,43.4
31,125.9,90.5
32,181.5,64.4
33,236.5,168.4
34,221.6,59.8
35,167.5,34.1
36,214.6,36.4
37,204.8,38.2
38,190.8,49.5
39,361.1,582.3
40,198.2,47.2
41,217.4,57.4
42,198.3,100.8
43,163.7,55.9
44,179.6,125.9
45,202.5,96.6
46,258.9,315.8
47,261.8,40.6
48,130.2,52.2
49,154.4,42.9
50,159.1,52.7
51,222,46.2
52,398.9,602.7
53,188,51.8
54,162.3,39.7
55,226.7,54.1
56,280,67.2
57,198.2,85.4
58,166.6,94.7
59,163.7,224.8
60,268.8,53.2
61,138.5,37.3
62,178.7,44.3
63,145.5,36.3
64,236.1,35
65,328.4,781.4
66,210.9,49
67,157.3,36.3
68,274.3,44.8
69,226.8,40.6
70,280.4,113.8
71,170.7,74.6
72,251.5,243.6
73,204.8,52.2
74,208.5,71.8
75,205.8,35
76,139.5,61.1
77,174.5,45.8
78,362,835.6
79,254.2,48.1
80,137.6,39.7
81,173.1,50.4
82,163.3,47.2
83,165.6,103.6
84,195.1,64.4
85,270.6,283.7
86,115.7,50.3
87,157.2,46.2
88,209.5,34.5
89,464.2,45.3
90,162.3,58.3
91,329.4,480.6
92,187.1,67.2
93,153.5,83.5
94,241.2,50.4
95,161.5,81.2
96,179.2,54.1
97,148.4,98
98,218.8,173.5
99,285,36.9
100,171.2,34.5
101,210.4,74.2
102,162.8,54.6
103,175.4,43.4
104,216.9,1006.4
105,166.1,54.1
106,142.3,38.8
107,199.6,42.5
108,140.4,44.8
109,248.2,59.7
110,139.9,100.3
111,221.6,275.8
112,201.1,51.3
113,141.8,69
114,155.8,37.3
115,202.5,49
116,153,45.8
117,420.3,544.5
118,250.5,48.5
119,133.4,53.6
120,249.6,52.3
121,164.2,45.2
122,261.7,83.9
123,251.9,72.8
124,261.8,235.2
125,174,56.5
126,126.9,35.9
127,189,39.2
128,164.2,50.8
129,173.1,69.5
130,670,
\end{filecontents*}
\pgfplotstableread[col sep = comma]{benchmark-tab.csv}\tableData
\pgfplotstableread[col sep = comma]{benchmark-graph.csv}\graphData
\pgfplotstableread[col sep = comma]{macro-benchmark-time.csv}\macroData

% Style to select only points from #1 to #2 (inclusive)
\pgfplotsset{select coords between index/.style 2 args={
    x filter/.code={
        \ifnum\coordindex<#1\def\pgfmathresult{}\fi
        \ifnum\coordindex>#2\def\pgfmathresult{}\fi
    }
}}