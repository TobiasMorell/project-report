% see, e.g., http://en.wikibooks.org/wiki/LaTeX/Customizing_LaTeX#New_commands
% for more information on how to create macros

%%%%%%%%%%%%%%%%%%%%%%%%%%%%%%%%%%%%%%%%%%%%%%%%
% Macros for the titlepage
%%%%%%%%%%%%%%%%%%%%%%%%%%%%%%%%%%%%%%%%%%%%%%%%
%Creates the aau titlepage
\newcommand{\aautitlepage}[3]{%
  {
    %set up various length
    \ifx\titlepageleftcolumnwidth\undefined
      \newlength{\titlepageleftcolumnwidth}
      \newlength{\titlepagerightcolumnwidth}
    \fi
    \setlength{\titlepageleftcolumnwidth}{0.5\textwidth-\tabcolsep}
    \setlength{\titlepagerightcolumnwidth}{\textwidth-2\tabcolsep-\titlepageleftcolumnwidth}
    %create title page
    \thispagestyle{empty}
    \noindent%
    \begin{tabular}{@{}ll@{}}
      \parbox{\titlepageleftcolumnwidth}{
        \iflanguage{danish}{%
          \includegraphics[width=\titlepageleftcolumnwidth]{images/aau-logo-da}
        }{%
          \includegraphics[width=\titlepageleftcolumnwidth]{images/aau-logo-en}
        }
      } &
      \parbox{\titlepagerightcolumnwidth}{\raggedleft\sf\small
        #2
      }\bigskip\\
       #1 &
      \parbox[t]{\titlepagerightcolumnwidth}{%
      \textbf{Abstract:}\bigskip\par
        \fbox{\parbox{\titlepagerightcolumnwidth-2\fboxsep-2\fboxrule}{%
          #3
        }}
      }\\
    \end{tabular}
    \vfill
    \iflanguage{danish}{%
      \noindent{\footnotesize\emph{Rapportens indhold er frit tilgængeligt, men offentliggørelse (med kildeangivelse) må kun ske efter aftale med forfatterne.}}
    }{%
      \noindent{\footnotesize\emph{The content of this report is freely available, but publication (with reference) may only be pursued due to agreement with the author.}}
    }
    \clearpage
  }
}

%Create english project info
\newcommand{\englishprojectinfo}[8]{%
  \parbox[t]{\titlepageleftcolumnwidth}{
    \textbf{Title:}\\ #1\bigskip\par
    \textbf{Theme:}\\ #2\bigskip\par
    \textbf{Project Period:}\\ #3\bigskip\par
    \textbf{Project Group:}\\ #4\bigskip\par
    \textbf{Participant(s):}\\ #5\bigskip\par
    \textbf{Supervisor(s):}\\ #6\bigskip\par
    \textbf{Copies:} #7\bigskip\par
    \textbf{Page Numbers:} \pageref{LastPage}\bigskip\par
    \textbf{Date of Completion:}\\ #8
  }
}

%Create danish project info
\newcommand{\danishprojectinfo}[8]{%
  \parbox[t]{\titlepageleftcolumnwidth}{
    \textbf{Titel:}\\ #1\bigskip\par
    \textbf{Tema:}\\ #2\bigskip\par
    \textbf{Projektperiode:}\\ #3\bigskip\par
    \textbf{Projektgruppe:}\\ #4\bigskip\par
    \textbf{Deltagere:}\\ #5\bigskip\par
    \textbf{Vejleder:}\\ #6\bigskip\par
    \textbf{Oplagstal:} #7\bigskip\par
    \textbf{Sidetal:} \pageref{LastPage}\bigskip\par
    \textbf{Afleveringsdato:}\\ #8
  }
}

%%%%%%%%%%%%%%%%%%%%%%%%%%%%%%%%%%%%%%%%%%%%%%%%
% An example environment
%%%%%%%%%%%%%%%%%%%%%%%%%%%%%%%%%%%%%%%%%%%%%%%%
\theoremheaderfont{\normalfont\bfseries}
\theorembodyfont{\normalfont}
\theoremstyle{break}
\def\theoremframecommand{{\color{aaublue!50}\vrule width 5pt \hspace{5pt}}}
\newshadedtheorem{exa}{Example}[chapter]
\newenvironment{example}[1]{%
		\begin{exa}[#1]
}{%
		\end{exa}
}

% Table macro.
% 1: Table Latex
% 2: Caption
% 3: Label
\definecolor{lightgray}{gray}{0.9}
\newcommand{\makeTable}[3]{
    \rowcolors{1}{}{lightgray}
    \begin{table}[H]
        \makebox[\textwidth][c]{
        \begin{tabular}
            #1
        \end{tabular}}
        \caption{#2}
        \label{tab:#3}
    \end{table}
    \rowcolors{0}{}{}
}

% Table macro.
% 1: Table Latex
% 2: Caption
% 3: Label
\newcommand{\makeTablePB}[3]{
    \centering
    \rowcolors{1}{}{lightgray}
    \begin{longtable}
        #1
        \caption{#2}
        \label{table:#3}
    \end{longtable}
    \rowcolors{0}{}{}
}

\newcommand{\tableref}[1]{Table \ref{tab:#1}}
\newcommand{\figureref}[1]{Figure \ref{fig:#1}}
\newcommand{\figref}[1]{Figure \ref{fig:#1}}
\newcommand{\coderef}[1]{Code \ref{code:#1}}
\newcommand{\lstref}[1]{Listing \ref{lst:#1}}
\newcommand{\secref}[1]{Section \ref{sec:#1}}
\newcommand{\chapref}[1]{Chapter \ref{chap:#1}}
\newcommand{\appendixref}[1]{Appendix \ref{app:#1}}
\newcommand{\lineref}[1]{Line \ref{line:#1}}

%\newcommand{\acrfull}[1]{\acrlong{#1} (\acrshort{#1})}

\newcommand{\dquote}[1]{``{#1}''}
\newcommand{\squote}[1]{`{#1}'}

\newcommand{\quoteWithCite}[3]{
  \begin{quotation}
  \noindent \textit{"#1"}
  \begin{flushright}
    -#2 \cite{#3}
  \end{flushright}
  \end{quotation}
}

\newcommand{\quoteWithoutCite}[2]{
  \begin{quotation}
  \noindent \textit{"#1"}
  \begin{flushright}
  -#2
  \end{flushright}
  \end{quotation}
}

\newcommand{\dvanbo}{\color{pink}}
\newcommand{\tmc}[1]{\todo[color=violet]{#1}}
\newcommand{\btc}[1]{\todo[color=red!75]{#1}}
\newcommand{\needcite}{\todo{Citation needed}}
\newcommand{\inline}[1]{\todo[inline]{#1}}
\newcommand{\ttt}[1]{\texttt{#1}}
\newcommand{\unity}{Unity}
\newcommand{\unityspace}{Unity }
\newcommand{\unreal}{Unreal Engine}
\newcommand{\unrealspace}{Unreal Engine }
\newcommand{\cryengine}{CryEngine}
\newcommand{\cryenginespace}{CryEngine }

%%%%%%%% Column types
\newcolumntype{L}[1]{>{\raggedright\let\newline\\\arraybackslash\hspace{0pt}}m{#1}}
\newcolumntype{C}[1]{>{\centering\let\newline\\\arraybackslash\hspace{0pt}}m{#1}}
\newcolumntype{R}[1]{>{\raggedleft\let\newline\\\arraybackslash\hspace{0pt}}m{#1}}
\newcolumntype{P}[1]{>{\raggedright\arraybackslash}p{#1}}

\newcommand{\benchmarkRes}[2]{
\begin{figure}[H]
    \makebox[\textwidth][c]{
    \begin{tikzpicture}
        \begin{axis}[
            ybar=0pt,
            label style={font=\bfseries},
            ylabel={Logarithmic Run Time (ns)},
            bar shift=0pt,
            width=1\textwidth,
            height=8cm,
            bar width=16pt,
            ymode=log,
            xtick=data,
            log basis y={2},
            ymajorgrids,
            xticklabel style={rotate=45, anchor=east},
            symbolic x coords={,
                Cry CSharp (editor),
                Cry C++ (editor),
                Cry C++ (standalone),
                Dotnet CSharp (debug),
                Dotnet CSharp (release),
                Godot CSharp (editor),
                Godot CSharp (release),
                Mono CSharp (debug),
                Mono CSharp (release),
                Unity CSharp (editor),
                Unity CSharp (release),
                Unreal C++ (editor),
                Visual C++ (debug),
                Visual C++ (release),
                GCC C++ (debug),
                GCC C++ (release),
                Arcadia},
        ]
            \addplot table [x=Engine, y={#1}] {\graphData};
        \end{axis}
    \end{tikzpicture}}
    \caption{#1 Test Results}
    \label{fig:#2}
\end{figure}
}

\newcommand{\benchmarkResOverview}[3]{
    \begin{figure}[H]
    \makebox[\textwidth][c]{
    \begin{tikzpicture}
    \tikzset{every mark/.append style={scale=.5}}
        \begin{axis}[
            ylabel={Logarithmic Run Time (ns)},
            width=\textwidth,
            height=8cm,
            ymode=log,
            ymajorgrids,
            xtick=data,
            log basis y={2},
            xticklabel style={rotate=45, anchor=east},
            symbolic x coords={
                Scale Vector 2D,
                Scale Vector 3D,
                Multiply Vector 2D,
                Multiply Vector 3D,
                Translate Vector 2D,
                Translate Vector 3D,
                Subtract Vector 2D,
                Subtract Vector 3D,
                Length Vector 2D,
                Length Vector 3D,
                Dot Product 2D,
                Dot Product 3D,
                Prime,
                Array Allocation,
                Sestoft
            },
            area legend,
            legend columns = -1,
            legend style={
                draw=none,
                at={(0.5,1.05)},
                anchor=south,
                column sep=1ex,
                font=\tiny
            }
        ]
        #3
        \end{axis}
    \end{tikzpicture}}
    \caption{#1}
    \label{fig:#2}
\end{figure}
}

\newcommand{\benchmarkResOverviewRoofed}[3]{
    \begin{figure}[H]
    \makebox[\textwidth][c]{
    \begin{tikzpicture}
    \tikzset{every mark/.append style={scale=.5}}
        \begin{axis}[
            ylabel={Logarithmic Run Time (ns)},
            width=\textwidth,
            height=8cm,
            ymode=log,
            ymajorgrids,
            xtick=data,
            ymax=250,
            log basis y={2},
            xticklabel style={rotate=45, anchor=east},
            symbolic x coords={
                Scale Vector 2D,
                Scale Vector 3D,
                Multiply Vector 2D,
                Multiply Vector 3D,
                Translate Vector 2D,
                Translate Vector 3D,
                Subtract Vector 2D,
                Subtract Vector 3D,
                Length Vector 2D,
                Length Vector 3D,
                Dot Product 2D,
                Dot Product 3D,
                Prime,
                Array Allocation,
                Sestoft
            },
            area legend,
            legend columns = -1,
            legend style={
                draw=none,
                at={(0.5,1.05)},
                anchor=south,
                column sep=1ex,
                font=\tiny
            }
        ]
        #3
        \end{axis}
    \end{tikzpicture}}
    \caption{#1}
    \label{fig:#2}
\end{figure}
}

\newcommand{\plotData}[2]{
    \addplot[mark=*, color=#2] table [x=Test, y={#1}] {\tableData};
    \addlegendentry{#1};
}
