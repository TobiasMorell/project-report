\section{Functional Game Engines} \label{sec:functional-game-engines}
Game development goes hand-in-hand with game engines. Thus it is natural to look for game engines that support the functional programming paradigm. This section presents three game frameworks that support functional languages; Helm \cite{helm:github}, Nu \cite{nu:github} and Arcadia \cite{arcadia:github}.

\subsection{Helm}
Helm is a game engine, written in and with support for Haskell \cite{helm:github,helm:wiki}. As such, Helm is purely functional. Helm uses the Elerea \ac{FRP} package along with a graphical framework called \ac{SDL} \cite{libsdl:about}. \ac{SDL} is written in C++, but is used with Haskell bindings in Helm.

Helm presents two concepts, that they deem to be a \textit{"powerful paradigm shift for game development"} \cite{helm:wiki}. The first concept is \textit{subscriptions}, which is used to map from keyboard events to game action types. A game action type is Helm's abstraction of an in-game action, such as moving a character. The other concept is called \textit{commands}. The wiki describes commands as \textit{"IO-like monads that have context about the engine state"}. It also states that commands can be mapped to game actions.

\subsection{Using Helm}
There is, to the best of our knowledge, currently only a single game that works with helm 1.0. The game is called \textit{Flappy} and is distributed with the engine \cite{hs-helm-flabby-shooter:github}. Apart from Flappy, there also exist a handful of programs that work with the older 0.7 version (which are not compatible with 1.0). Furthermore the documentation is sparse in working examples, which makes it difficult to get started. The best bet is copying Flappy, which sums up to around 470 lines of both engine- and gameplay code, and modifying it.

To become more familiar with the Helm engine we attempted to extend Flappy. The core of this change was to add a Subscription to a shoot button. In this case the shoot button is the "Z" key. This function is displayed in \lstref{haskell:shoot}. In addition, a function to check for collision between the bullet and the obstacles was also added, as the engine does not provide collision detection.

\begin{lstlisting}[style=haskell-highlight,label={lst:haskell:shoot},caption={Haskell Shoot}]
update model@Model { .. } Shoot =
  if playerStatus == Dead || playerStatus == Waiting
  then (model, Cmd.none)
  else
      if canFireBullet bulletPos flapperPos
      then
        ( model
          { bulletPos   = flapperPos 
          ,  playerStatus = Playing
          }
        , Cmd.none
        )
else (model, Cmd.none)
\end{lstlisting}

It turned out that crafting this small addition to the game boiled down to mere guesswork on how to do so. Part of the reason for this is that there is a very small community around Helm, a complete lack of useful examples and no documentation. We therefore choose to not examine Helm in further detail.

\subsection{Nu}
Nu is a 2D game engine written in F\# and C\#. It uses \ac{SDL} for rendering, Farseer for physics, Tiled\# for managing the game world and Gaia as game editor \cite{nu:github}. It is still under development and is currently missing support for 3D games, particle systems and AI.

Unlike Haskell, F\# is not strictly pure. Bryan Edds, the developer of Nu, has made modifications to F\#'s type system, that makes it default to pure. Developers may still opt out of F\#'s purity and for instance declare mutable variables \cite{nu:pdf}.

\subsection{Using Nu}
As of the time of writing, Nu is only implemented on Windows and requires Visual Studio. Nu can be setup by cloning the github project and installing a special bash file, such that Visual Studio can recognise Nu's special version of F\#. Afterwards the Nu engine may be opened in Visual Studio and projects can be added to the Nu engine.

\begin{lstlisting}[language=fsharp,label={lst:FSharp:Bullet},caption={F\# BulletDispatcher, as found in the BlazeVector game \cite{nu:blazevector}}]
type BulletDispatcher () =
        inherit EntityDispatcher ()

        static let [<Literal>] BulletLifetime = 27L

        static let handleUpdate evt world =
            let bullet = evt.Subscriber : Entity
            let world = bullet.SetAge (inc (bullet.GetAge world)) world
            if bullet.GetAge world > BulletLifetime
            then World.destroyEntity bullet world
            else world

        static let handleCollision evt world =
            let bullet = evt.Subscriber : Entity
            if World.isTicking world
            then World.destroyEntity bullet world
            else world

        static member PropertyDefinitions =
            [Define? Size (Vector2 (20.0f, 20.0f))
             Define? Omnipresent true
             Define? Density 0.25f
             Define? Restitution 0.5f
             Define? LinearDamping 0.0f
             Define? GravityScale 0.0f
             Define? IsBullet true
             Define? CollisionBody (BodyCircle { Radius = 0.5f; Center = Vector2.Zero })
             Define? StaticImage Assets.PlayerBulletImage
             Define? Age 0L]

        static member IntrinsicFacetNames =
            [typeof<RigidBodyFacet>.Name
             typeof<StaticSpriteFacet>.Name]

        override dispatcher.Register (bullet, world) =
            let world = World.monitor handleUpdate (Events.Update ->- bullet) bullet world
            let world = World.monitor handleCollision (Events.Collision ->- bullet) bullet world
            world
\end{lstlisting}
\lstref{FSharp:Bullet} is taken from the BlazeVector example project by the developer of Nu \cite{nu:blazevector}. It defines the behaviour of a bullet, that gets shot.
Because of the lacking documentation the only source of help was well commented but highly abstracted finished game projects. 
Understanding and working in Nu became a confusing struggle.

\subsection{Arcadia}
Arcadia takes a different approach to functional game development. Instead of providing a full-fledged game engine, they have integrated Clojure into \unityspace\cite{arcadia:github,arcadia:usage}, as a interpreted language. In order to do so they have used an implementation of Clojure in Microsoft's \ac{CLR}\cite{arcadia:clr}. This implementation allows Clojure to run on the .NET \ac{VM} instead of its default \ac{JVM}.

Clojure is primarily a pure functional language, but it allows the use of imperative constructs, for instance the \texttt{set!}-function\cite{arcadia:usage}. Furthermore, \unityspace is \textit{"a highly mutable, stateful system"}, with which argument the developers of Arcadia allow some functions to have side-effects. This makes sense since Clojure supports \ac{STM} \cite{clojure2018transactions} which can be used to handle race conditions created by the side effects. Their example is the \texttt{LookAt}-function in the \texttt{Camera}-class, which will mutate the camera's rotation to point it in the direction of the point, which is passed as argument to the function. Their argumentation of allowing this kind of behaviour is, that they wish to be conservative with imposing their own ideas on \unity's \ac{API}.

\subsection{Using Arcadia}
To use Arcadia, the game engine \unityspace has to be installed first. Then Arcadia is cloned using git into a \unityspace project \cite{arcadia-test:github}.
From here, it's possible to use the \ac{REPL} to bind functions to \unity's triggers and events.
To test Arcadia out, we attempted to create the same shooting function as with the Helm example, but to an otherwise empty game.

\begin{lstlisting}[language=clojure,label={lst:clojure:shoot},caption={Arcadia Shoot}]
(defn shoot []
  (let [proj (instantiate
               (Resources/Load "Projectile"))] 
    (with-cmpt proj [tr Transform] 
      (set! (.position tr) (.position (cmpt (get-player) UnityEngine.Transform)))
      (roles+ proj
              (-> projectile-roles
                  (assoc-in [::lifespan :state :start] System.DateTime/Now)
                  (assoc-in [::lifespan :state :lifespan] 5000)))
      proj)))
\end{lstlisting}

The core of this functionality is binding a behaviour to a player-object, we call this behaviour \texttt{move-when-arrow-pressed}, which moves the player. In addition to moving the player, it also checks every frame, whether spacebar is pressed, in which case the shoot function is called. 
This function is displayed in \lstref{clojure:shoot}.

\begin{lstlisting}[language=clojure,label={lst:clojure:repl},caption={Arcadia Shoot}]
(hook+ (get-player) :update #'minimal.mmo/move-when-arrow-pressed)
\end{lstlisting}

\texttt{move-when-arrow-pressed} should be called on every frame of the game. It is therefore bound to the player-object's update trigger.
To bind the function to the player-object, we issue the command displayed in \lstref{clojure:repl} to the Arcadia REPL. Here, \texttt{minimal.mmo} is simply our namespace.

\subsection{Paradigm Approach} \label{sec:para-approach} \tmc{Change this title}
In this section we presented three different strategies when it comes to functional languages for game development. Helm is a pure functional game engine that uses Haskell. Nu is an engine written F\#, which defaults to pure functional, but allows imperative constructs. Finally Arcadia integrates Clojure into Unity. In the beginning of this report we presented a quotation from Tim Sweeney, who pointed out that purely functional is right default and that imperative constructs were vital, but should be exposed through explicit effects-typing constructs. In his talk he called out for a new language, but nonetheless, this mantra is well supported in all three approaches presented here; Haskell support imperative behaviour through Monads, F\# supports impurity using the \texttt{let mutable}-keyword and Clojure using the \ttt{!}-operators.