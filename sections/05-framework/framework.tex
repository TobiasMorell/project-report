\chapter{Framework}
In this chapter a framework for engine or language profiling is introduced. In this context a profile is a set of metrics that describe the strengths and weaknesses of an engine or a programming language. The reason that the same framework can be used to profile languages and engines is that an engines \ac{API} usually affects languages use to such a degree that it may be considered a separate language.

\section{Profiles}
The profiles generated by this framework cover three important domains of a game engine or development language: \ac{CPU} utilisation, \ac{GPU} utilisation and the usability of the game engine/language. The \ac{CPU} utilisation is measured using a set of benchmarks that emulate high workload of a running game. The \ac{GPU} is not covered by the first version of the framework. The usability of the engine/language is a measure of the difficulty of learning and using the system. This is estimated using a static-analysis method called cognitive dimensions (see \secref{cog-dim})\todo{Update if method is changed}.

\subsection{Metrics}
Each domain is examined using a number of metrics. These metrics vary for each domain and are not necessarily comparable across domains or even inside domains, as is the case with the usability domain. The metrics for each domain is listed below. According to E. F. Anderson \cite{5962102} a game engine must consist of the subsystems \figureref{engine-subsystes}. These subsystems are considered dimensions and are tested using different tests

\begin{description}
    \item[\textbf{CPU Utilisation}] concerns itself with the performance of the engine/language. Some engines make useful abstractions available at the cost of performance and other engines are so performant that they outperform general programming languages. This is what we call \ac{CPU} utilisation.
    \begin{itemize}
        \item Physics
        \item Artificial intelligence
        \item Scripting (Behaviour)
    \end{itemize}
    \item[\textbf{Usability}] is a measure of how difficult a language is to learn and use. A number of relevant metrics have been selected from the cognitive dimensions methodology to be used here as metrics.
    \begin{itemize}
        \item Diffuseness/Terseness
        \item Hidden Dependencies
        \item Premature Commitment
        \item Viscosity 
    \end{itemize}
    \item[\textbf{I/O Utilisation}] is the engine/languages handling of \ac{I/O}. Here the efficacy is considered as well as the accessibility of the \ac{I/O} layer. Accessibility in this case is the range of available \ac{I/O} operations.
    \begin{itemize}
        \item Input
        \item Networking
    \end{itemize}
    \item[\textbf{GPU Utilisation}] is a measure of \ac{GPU} utilisation. 
    \begin{itemize}
        \item Rendering
        \item Lighting?
    \end{itemize}
\end{description}

% \item Micro-benchmarks
%         \begin{itemize}
%             \item 2D Vectors
%             \item 3D Vectors
%             \item Math
%             \item Memory
%         \end{itemize}
%         \item Macro-benchmarks
%         \begin{itemize}
%             \item Wombats
%         \end{itemize}