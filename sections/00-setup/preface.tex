\chapter*{Preface\markboth{Acknowledgements}{Acknowledgements}}\label{ch:preface}
\addcontentsline{toc}{chapter}{Preface}

\section*{Literature Search}
As large parts of this project revolves around literature research, a systematic set of guidelines on how topics are explored may prove useful both to the project group and the reader. To the reader it may be used to confirm (or reject) that the topics discussed in this paper have actually been thoroughly explored.

First and foremost a set of criteria defining a relevant article is required. The most important and first criteria is that the article should be published within the field of Computer Science. Secondly the article should either be up to date or have had some influence. This is vague criteria and will boil down to preference, but a ten year old article with zero or just a few citations is unlikely to provide relevant material. Articles that have been published within the last two to three years are, on the other hand, likely to have few citations due to their age, and may provide relevant content.

Search engines are the primary tool in literature search and therefore of utmost importance. In the literature search in this project we will use three engines; DBLP, \ac{ACM}'s Digital Library and Google Scholar. 
Queries go hand-in-hand with search engines. In case the research ever reaches a dead-end, the queries used to explore the topic is listed in the Appendix along with the results from said queries. Note that some queries may actually yield results, that are discarded based on the criteria listed above.
Similarly a good strategy in literature research is to examine which article has cited a particular paper to see what has happened since the article was published. Both \ac{ACM}'s Digital Library and Google scholar support this at the click of a button. The results from this strategy will also be evaluated against the criteria.

As the game development field is often hidden behind closed doors, sometimes we use non-scholarly sources. 
These sources are taken with a grain of salt as they may not be entirely true.
Even so, they will say something about the specific topic.
We take into consideration what the claim is, who is stating it, and whether it aligns with current knowledge. 