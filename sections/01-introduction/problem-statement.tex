\section{Problem Statement}
In this section we will formulate a problem that this project will attempt to solve. The problem will be expressed as an exploratory question and a set of sub-questions. These will direct the research of this project throughout the remainder of this report. Research will be divided into the \ac{SotA} section and the wider background research section.

The research of the \ac{SotA} will focus on the game development and how it is currently conducted. The tools used are examined in particular, here the tools are programming languages and game engines. Thus the focal point is which languages are used and what are their properties, as well as which game engines are sued. Furthermore, understanding why game engines are used is examined as well.

In the background research section the focus will be on the wider domain. Here related papers, methodology and anything else relevant to the field and this project is examined. Tools used in game development and game development literature, which is not covered in \ac{SotA}, is covered here. Furthermore, due to the number of available tools, we need some methodology to formally compare available tools. Therefore some comparison methodology will also be covered here.

Furthermore, we have found that engines are so complex and restrictive, that they practically are a separate language. This hypothesis is assumed going forward in the report. Given the inquiry documented above, we pose the following problem statement:

\begin{center}
    \begin{enumerate}
        \item \textbf{How can game engines/languages be compared?}
        \item How can performance engine/language be reliably measured?
        \item Are game engines comparable with programming languages?
        \item What metrics are most relevant for comparison?
        \item Which metrics of engines/languages are equivalent and which are not?
    \end{enumerate}
\end{center}